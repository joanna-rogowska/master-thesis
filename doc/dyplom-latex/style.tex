%*******************************************************************************
% Definicje stylu dokumentu
%*******************************************************************************

%===============================================================================
% klasa dokumentu

\documentclass[12pt, a4paper, oneside, titlepage, final]{book}
%\documentclass[12pt,a4paper,onecolumn,oneside,11pt,wide,floatssmall]{book}


%===============================================================================
% Pakiety
%\usepackage[latin2]{inputenc}
\usepackage[cp1250]{inputenc}
%\usepackage[utf8]{inputenc}				% kodowanie �r�d�a
\usepackage[polish]{babel}				% polskie przenoszenie wyraz�w (hyph.)
\usepackage[OT4]{fontenc}					% font PL
\usepackage{url}								% polecenie \url
\usepackage{amsfonts}						% fonty matematyczne
\usepackage{graphicx}						% wstawianie grafiki
\usepackage{listings}						% wstawianie kodu
\usepackage{color}							% kolory
\usepackage{fancyhdr}						% paginy g�rne i dolne
\usepackage[plainpages=false]{hyperref}% dynamiczne linki
\usepackage{calc}								% operacje arytmetyczne w TeX'u
\usepackage{tabularx}						% rozci�gliwe tabele
\usepackage{array}							% standardowe tabele
\usepackage{hyperref}
\linespread{1.3}								% 1.3 do interlinii 1.5


% w�asne pakiety
\usepackage{pwtitle}

%===============================================================================
% Ustawienia dokumentu

\frenchspacing

% ustawienia wymiar�w
\oddsidemargin 20mm							% margines nieparzystych stron
\evensidemargin 20mm							% margines parzystych stron
\headheight 15pt								% wysoko�� paginy g�rnej
\topmargin 0mm									% margines g�rny

% styl paginacji
\pagestyle{fancy}
\renewcommand{\chaptermark}[1]{\markboth{#1}{}}
\renewcommand{\sectionmark}[1]{\markright{\thesection\ #1}{}}

% nag��wek 
\fancyhf{}
\fancyhead[L,RO]{\thepage}
\fancyhead[LO]{\small\nouppercase{\rightmark}}
%\fancyhead[R]{\small\nouppercase{\leftmark}}
\renewcommand{\headrulewidth}{0.1pt}
\renewcommand{\footrulewidth}{0pt}

% nag��wek w stylu plain 
\fancypagestyle{plain}
{
\fancyhf{}
\renewcommand{\headrulewidth}{0pt}
\renewcommand{\footrulewidth}{0pt}
}

% ta sekwencja tworzy czyste kartki na stronach po \cleardoublepage
\makeatletter
\def\cleardoublepage{\clearpage\if@twoside \ifodd\c@page\else
	\hbox{}
	\vspace*{\fill}
	\thispagestyle{empty}
	\newpage
	\if@twocolumn\hbox{}\newpage\fi\fi\fi}
\makeatother

%===============================================================================
% Zmienne �rodowiskowe i polecenia

% definicja
\newtheorem{definition}{Definicja}[chapter]

% twierdzenie
\newtheorem{theorem}{Twierdzenie}[chapter]

% obcoj�zyczne nazwy
\newcommand{\foreign}[1]{\emph{#1}}

% pozioma linia
\newcommand{\horline}{\noindent\rule{\textwidth}{0.4mm}}

% wstawianie obrazk�w {plik}{caption}{opis}
\newcommand{\fig}[3]
{
\begin{figure}[!htb]
\begin{center}
\includegraphics[width=\textwidth]{#1}
\caption[#2]{#2. #3}
\label{#1}
\end{center}
\end{figure}
}

%===============================================================================
% ustawienia pakietu hyperref

\hypersetup
{
%colorlinks=true,			% false: boxed links; true: colored links
%linkcolor=black,			% color of internal links
%citecolor=black,			% color of links to bibliography
%filecolor=black,			% color of file links
%urlcolor=black			% color of external links
}

%===============================================================================
% ustawienia pakietu listings

% kolor listingu
\definecolor{ListingBackground}{rgb}{0.97,0.97,0.97}

% ustawienia otoczenia
\lstset
{
language=C++,                   % choose the language of the code
basicstyle=\footnotesize,       % the size of the fonts
numbers=left,                   % where to put the line-numbers
numberstyle=\footnotesize,      % the size of the fonts for the line-numbers
stepnumber=1,                   % the step between two line-numbers.
numbersep=8pt,                  % how far the line-numbers are from the code
backgroundcolor=\color{white},  % choose the background color.
showspaces=false,               % show spaces adding particular underscores
showstringspaces=false,         % underline spaces within strings
showtabs=false,                 % show tabs within strings
frame=none,                     % adds a frame around the code
tabsize=3,                      % sets default tabsize to 2 spaces
captionpos=b,                   % sets the caption-position to bottom
breaklines=true,                % sets automatic line breaking
breakatwhitespace=false,        % sets if automatic breaks only at whitespace
escapeinside={(*@}{@*)},        % if you want to add a comment within your code
xleftmargin=1cm,
xrightmargin=0.5cm,
keywordstyle={\bf\footnotesize\color{blue}},
backgroundcolor={\color{ListingBackground}},
framexleftmargin=3pt,
framexbottommargin=3pt,
framextopmargin=3pt,
framexrightmargin=3pt,
frame=tb,
framerule=0.1pt
}

%===============================================================================
