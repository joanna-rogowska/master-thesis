%*******************************************************************************
% Rozdzia� pierwszy
%*******************************************************************************

\chapter{Wst�p}
\label{cha:wprowadzenie}

jak powinna dzia�a� wyszukiwarka:
- znajdowa� obrazy podobne anatomicznie niezale�nie od modalno�ci czy w pierwszej kolejno�ci pokazywa� podobne w danej modalno�ci?\\
Za pokazywanie danej modalono�ci przemawia fakt, �e ka�da modalno�� odpowiada za znalezienie obraz�w o podobnych patologiach (fizjologia) lub obrazowa� anatomicznie. Np. Je�eli zapytanie obrazem PET - typowo funkcjonalnym to nie oczekujemy znalezienia danego organu anatomicznego tylko obrazy podobne w obr�bie modalno�ci
%###############################################################################
\section{Cel pracy}
\label{sec:cel_pracy}
- skonstruowanie nowoczesnej wyszukiwarki obraz�w medycznych z intuicyjnym gui \\
(rozdzia� 2 ma za zadanie wykaza�, �e jest z tym spory problem)\\
- znalezienie obraz�w o podobnej modalno�ci \\
(rozdz. 2 ma wykaza�, �e zwykle nie s� rozr�niane modalno�ci obrazowania, a wyniki nie zawsze s� trafione)\\
nie jest potrzebna specjalistyczna wiedza diagnostyczna (radiologiczna?) do rozwi�zania tego zagadnienia\\

