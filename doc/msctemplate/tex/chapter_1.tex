\chapter{Wstęp}

Rozdział ten poświęcony jest opisowi aktualnie używanych metod w biometrii służących do 
identyfikacji poprzez porównywanie odcisków palców, oraz sposobów pozyskiwania obrazów odcisków.

\section[Sposoby pozyskiwania obrazów odcisków][Sposoby pozyskiwania obrazów odcisków]{Sposoby pozyskiwania obrazów odcisków}

Aby można było mówić o identyfikacji użytkownika poprzez odcisk palca obraz taki należy od niego pobrać. Istnieje kilka metod pobierania odcisków. Niektóre z nich są używane powszechnie, inne to albo specyficzne sposoby dla policji lub testowane nowinki techniczne. Ciężko jest je klasyfikować pod względem lepszy-gorszy, Każdy z przedstawianych sposobów jest przeznaczony do innych celów i powinien być stosowany w zależności od przeznaczenia pobranej próbki.

\vspace{.5cm}\par
Sposoby pobierania odcisku można klasyfikować poprzez sposób przykładania palca do sensora. Mówiąc sensor, chodzi o dostarczone urządzenie zapisujące odcisk. Może to być zarówno czytnik optyczny jak i atrament z kartką papieru. Wyróżniono trzy metody:

\renewcommand*{\labelitemi}{\bullet}
\begin{itemize}
	\item Rolowanie (ang. Rolled finger)
	\item Pobór statyczny (ang. Static sensing)
	\item Pobór poprzez poślizg (ang. Sweep (swipe) reading)
\end{itemize}
\vspace{.5cm}\par

\textbf{Rolowanie}, sposób aktualnie używany głównie przez służby policyjne, tylko w połączeniu z kartką i atramentem, Jest to jeden z pierwszych sposobów pobierania odcisków.\footnote{Oczywiście istnieją starsze sposoby pozyskiwania odcisków, jak choćby stosowane w starożytności gliniane tabliczki z odciskiem dłoni pomagające identyfikować kupców} Główną zaletą takiego sposobu jest pozyskanie całego odcisku od "brzegu do brzegu", ma to sprzyjać maksymalizacji prawidłowych rozpoznań podejrzanych. Policja, jak też inne służby mundurowenie nie polega jedynie na komercyjnych systemach biometrycznych. Zamiast tego mają sztab specjalistów rozpoznających odciski po najdrobniejszych szczegółach. Często muszą badać odciski utajone, czyli takie pozostawione niejawnie np. na szklance lub innym przedmiocie zbrodni. Dlatego ważne jest aby pozyskać jak największą część odcisku. Metoda ta nie ma najmniejszych szans komercyjnego zastosowania, gdyż niesie za sobą silne skojarzenia kryminalne i źle kojarzy się społeczeństwu. 
\vspace{.5cm}\par

\textbf{Pobór statyczny}, jedna z obecnie najpopularniejszych metod poboru odcisków. Użytkownik musi wyłącznie położyć palec w wyznaczonym miejscu i nie ruszać nim do momentu pobrania próbki. Rozwiązanie jest na tyle proste, że użytkownik nie potrzebuje żadnego przeszkolenia aby samodzielnie go używać. Niestety rozwiązanie to ma kilka wad. Po pierwsze nie jest oczywiste z jaką siłą należy nacisnąć na sensor. Silniejszy nacisk spowoduje dokładniejsze pobranie próbki. Jeżeli pobranie odcisku trwa zbyt długo, użytkownik ma naturalną tendencję do niecierpliwienia się, oraz niepotrzebnego naciskania na sensor. Zbyt silny nacisk ,oże w przyszłości skutkować uszkodzeniem urządzenia. Niemożliwe jest również utrzymanie palca w całkowitym bezruchu, co w oczywisty sposób powoduje niekorzystne zniekształcenia obrazu. Dodatkowo sensor po niedługim czasie używania staje się brudny, a zabrudzenia są rejestrowane jako fragmenty odcisku. Najważniejszą wadą rozwiązania jest pozostawianie własnego odcisku na samym sensorze. W przypadku gdy użytkownik stosuje odcisk palca jako klucz do systemu zabezpieczeń, zostawia wszystkie dane niezbędne do włamania na czytniku.
\vspace{.5cm}\par

\textbf{Pobór poprzez poślizg} jedna z nowszych metod. Używana głównie w zabezpieczaniu urządzeń takich jak laptop, tablet czy telefon. Pomimo iż urządzenia tego typu są pełne odcisków użytkownika sposób pobrania próbki nieco utrudnia powielenie odcisku i zaaplikowanie go na czytniku. Rozwiązanie to również nie jest pozbawione wad. Przede wszystkim nie jest to naturalny sposób pozostawiania odcisku, czyli osoby korzystające z takiego czytnika muszą najpierw nauczyć się go obsługiwać. Zaletą tego rozwiązania jest to iż użytkownik nie zostawia na samym czytniku swojego odcisku. Jeżeli taki sposób pobierania odcisku zamontowano by np w bankomatach oszuści nie byli by w stanie ukraść naszych odcisków. Nieco inaczej sprawa ma się z naszymi prywatnymi urządzeniami. Dodatkowo czytnik jest czysty, każde pobranie odcisku czyści sensor, w ten sposób eliminuje się zakłócenia obrazu. 
\vspace{.5cm}\par

Sposoby pobierania odcisku można również podzielić ze względu na zastosowaną technologię. 
\renewcommand*{\labelitemi}{\bullet}
\begin{itemize}
	\item Atrament i kartka
	\item Czytnik optyczny
	\item Czytnik optyczno elektryczny
	\item Czytnik pojemnościowy
	\item Czytnik naciskowy
	\item Czytnik ultrasonograficzny
\end{itemize}
\vspace{.5cm}\par

Czytniki optyczne wykorzystują zasadę całkowitego wewnętrznego odbicia światła. Użytkownik przykłada palec do szklanej powierzchni, która podświetlana od spodu przez lampę. W klasycznym rozwiązaniu użytkownik przykłada palec do podświetlanej szybki. W miejscu, w którym grzbiety linii papilarnych dotykają czytnika światło jest pochłaniane, pozostały odbity obraz służy do rekonstrukcji odczytanego odcisku. Istnieją również wersje bezdotykowe, w których palec jest oświetlany od spodu, jak też wersje do pobierania odcisku przez poślizg.

Czytniki optyczno elektryczne wykorzystują do swojego działania polimery, które są w stanie emitować światło, gdy są pobudzane odpowiednim napięciem. Czytnik emituje światło w miejscu gdzie grzbiety linii papilarnych go dotykają. Ponieważ czytnik połączony jest z kamerą CMOS, w łatwy sposób można odtworzyć obraz odcisku.

Czytniki pojemnościowe i naciskowe działają na podobnej zasadzie i wykorzystują nierówności palca wynikające z istnienia grzbietów i dolin linii papilarnych. W pierwszym wariancie w miejscu gdzie palec nie dotyka do czytnika tworzy się ''kondensator'' o mierzalnej pojemności, w przypadku drugim badamy miejsca gdzie grzbiety naciskają na czytnik.

Czytniki ultrasonograficzne są nowinką techniczną w sposobie pobierania odcisków. Niestety urządzenia te są dość dużych rozmiarów i są drogie. Dodatkowo pobranie odcisku zajmuje więcej czasu niż w przedstawianych powyżej metodach. Na razie wydają się być nieodpowiednie dla komercyjnej produkcji.

\section[Dotychczasowe metody identyfikacji][Dotychczasowe metody identyfikacji]{Dotychczasowe metody identyfikacji}

\subsection[Metody minucyjne][Metody minucyjne]{Metody minucyjne}

Metody minucyjne są obecnie jednymi z najczęściej stosowanych metod porównywania odcisków palców. Wynika to z ich prostoty opisu podobieństwa, a także akceptacji tego sposobu przez prawo w większości krajów. Metoda porównuje zebrana próbkę biometryczną z zapisanym w bazie wzorcem. Można to nazwać relacją wzorzec-próbka, gdzie jeden lub więcej odcisków stanowi bazę wiedzy, próbka to odcisk palca który chcemy porównać z wzorcem. Metody minucyjne szukają jak największej liczby zgodnych minucji i na jej podstawie orzekają podobieństwo bądź niepodobieństwo, nie jest brana pod uwagę procentowa zgodność, a jedynie liczba minucji. Metody te nie biorą pod uwagę kształtu ani układu linii papilarnych. W powszechnie stosowanych systemach liczba niezgodnych minucji nie jest brana pod uwagę. Systemy tak naprawdę szukają jedynie podobieństw. Jest to jedna z pierwszych metod porównywania odcisków i jednocześnie najskuteczniejsza. 
\subsection[Metody nieminucyjne][Metody nieminucyjne]{Metody nieminucyjne}

Pomimo wysokiej skuteczności metod minucyjnych mają one kilka wad dla których warto jest szukać metod nieminucyjnych.
Po pierwsze wiarygodne wyznaczenie minucji z obrazu o niskiej jakości jest bardzo trudne, a czasem wręcz niemożliwe. Ekstrakcja minucji jest kosztowna. Pomimo iż postęp obliczeniowy jest ogromny i być może doczekamy się czasów gdy ilość pamięci RAM nie będzie praktyczni ograniczona, to na razie czas obliczeń jest istotnym problemem. Metody nieminucyjne powinny być ''lżejsze'' obliczeniowo. Metody nieminucyjne wcale nie muszą być używane jako zastępnik metod minucyjnych, a jedynie jako metoda wspomagająca. Być może rozwiązanie takie przyczyni się do zwiększenia skuteczności porównywania odcisków. 

Metody te opierają się na porównywaniu innych cech odcisków. Należą do nich:
\renewcommand*{\labelitemi}{\bullet}
\begin{itemize}
	\item Rozmiar, kształt i ogólna sylwetka odcisku
	\item Liczba, typ i pozycja punktów osobliwych odcisku
	\item Relacje przestrzenne między liniami papilarnymi
	\item Rozkład, położenie i liczba porów
\end{itemize}
\vspace{.5cm}\par

\section[Cel Pracy][Cel Pracy]{Cel Pracy}

Celem pracy jest zbadanie możliwości odejścia od obiektowego modelu reprezentacji minucji w porównywaniu odcisków palców. Należy również zdefiniować podobieństwo i niepodobieństwo odcisków.
Należy zbadać możliwość stworzenia kodu odcisku, zaprogramować narzędzia umożliwiające porównywanie odcisków poprzez jego kod. Porównywania powinny odbywać się pod względem podobieństwa, niepodobieństwa oraz obu tych cech jednocześnie. Jest to o tyle ważne iż obecnie stosowane metody pomijają aspekt niepodobieństwa obrazów. Należy przeprowadzić analizę otrzymanych wyników
i porównać je na tle komercyjnego rozwiązania. Celem pracy nie jest stworzenie kolejnego oprogramowania do porównywania odcisków. Nie jest to zadanie łatwe, a tym bardziej nie jest przewidziane dla jednej osoby.

