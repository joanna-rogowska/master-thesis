
\begin{titlepage}
    % Strona tytułowa
    \vbox to\textheight{\hyphenpenalty=10000
    \begin{center}
	\begin{tabular}{p{107mm} p{9cm}}
	    \begin{minipage}{9cm}
	      \begin{center}
	      Politechnika Warszawska \\
	      Wydział Elektroniki i Technik Informacyjnych \\
	      Instytut Automatyki i Informatyki Stosowanej
	      \end{center}
	    \end{minipage}
	    &
	    \begin{minipage}{8cm}
	    \begin{flushleft}
	     \footnotesize
	      Rok akademicki 2012/2013
	    \vspace*{2.75\baselineskip}
	    \end{flushleft}
	    \end{minipage} \\
	\end{tabular}
	\vspace*{3.75\baselineskip}
	\par\vspace{\smallskipamount}
	\vspace*{2\baselineskip}{\LARGE Praca dyplomowa inżynierska\par}
	\vspace{3\baselineskip}{\LARGE\strut Cezary Guz\par}
	\vspace*{2\baselineskip}{\huge\bfseries Odległość, podobieństwo i niepodobieństwo obrazów na przykładzie odcisków palców\par}

	\vspace*{7\baselineskip}
	\hfill\mbox{}\par\vspace*{\baselineskip}\noindent
	\begin{tabular}[b]{@{}p{3cm}@{\ }l@{}}
	    {\large\hfill } & {\large }
	\end{tabular}
	\hfill
	\begin{tabular}[b]{@{}l@{}}
	Opiekun pracy: \\[\smallskipamount]
	{\large prof. dr hab. Andrzej Pacut}
	\end{tabular}\par
	\vspace*{4\baselineskip}
    \begin{tabular}{p{\textwidth}}
    \begin{flushleft}
	\begin{minipage}{7cm}
	Ocena \dotfill
	\par\vspace{1.6\baselineskip}
	\dotfill
	\par\noindent
	\centerline{\footnotesize Podpis Przewodniczącego} \par
	\centerline{\footnotesize Komisji Egzaminu Dyplomowego}\par
	\end{minipage}
    \end{flushleft}
    \end{tabular}
    \end{center}}

    % Życiorys
    \newpage\thispagestyle{empty}
    \begin{tabular}{p{5cm} p{12cm}}
    \begin{minipage}{5cm}
    \center
    \includegraphics[height=6.5cm,width=4.5cm]{img/foto.jpg}
    \end{minipage}
    &
    \begin{minipage}{12cm}
    \begin{flushleft}
    \par\noindent\vspace{1\baselineskip}
    \begin{tabular}[h]{l l}
    {\normalsize\it Specjalność:} & Informatyka -- \\
    & Systemy informacyjno decyzyjne
    \end{tabular}
    \par\noindent\vspace{1\baselineskip}
    \begin{tabular}[h]{l l}
    {\normalsize\it Data urodzenia:} & {\normalsize 8 stycznia 1989 r.}
    \end{tabular}
    \par\noindent\vspace{1\baselineskip}
    \begin{tabular}[h]{l l}
    {\normalsize\it Data rozpoczęcia studiów:} & {\normalsize 1 października 2008 r.}
    \end{tabular}
    \par\noindent\vspace{1\baselineskip}
    \end{flushleft}
    \end{minipage}
    \end{tabular}
    \vspace*{1\baselineskip}
    \begin{center}
	{\large\bfseries Życiorys}\par\bigskip
    \end{center}

    \indent
    Nazywam się  Guz Cezary urodziłem się 8 stycznia 1989r. w Lubartowie. W 2005 roku ukończyłem Publiczne gimnazjum Nr 4 im Komisji Edukacji Narodowej w Puławach. Swoją edukację następnie kontynuowałem w II liceum ogólnokształcącym im. Komisji Edukacji Narodowej w Puławach, w klasie o profilu matematyczno-fizyczno-informatycznym. W październiku 2008 roku rozpocząłem studia na wydziale Elektroniki i Technik Informacyjnych Politechniki Warszawskiej na kierunku Informatyka.
    \par
    \vspace{2\baselineskip}
    \hfill\parbox{15em}{{\small\dotfill}\\[-.3ex]
    \centerline{\footnotesize podpis studenta}}\par
    \vspace{3\baselineskip}
    \begin{center}
 	{\large\bfseries Egzamin dyplomowy} \par\bigskip\bigskip
    \end{center}
    \par\noindent\vspace{1.5\baselineskip}
    Złożył egzamin dyplomowy w dn. \dotfill
    \par\noindent\vspace{1.5\baselineskip}
    Z wynikiem \dotfill
    \par\noindent\vspace{1.5\baselineskip}
    Ogólny wynik studiów \dotfill
    \par\noindent\vspace{1.5\baselineskip}
    Dodatkowe wnioski i uwagi Komisji \dotfill
    \par\noindent\vspace{1.5\baselineskip}
    \dotfill

    % Streszczenie
    \newpage\thispagestyle{empty}
    \vspace*{2\baselineskip}
    \begin{center}
	{\large\bfseries Streszczenie}\par\bigskip
    \end{center}

    {\itshape
    Praca ta opisuje implementację i wyniki różnych wariantów algorytmu kodującego. Przedstawia wady i zalety takiego podejścia. Praca nie ma na celu stworzenia kolejnego oprogramowania do porównywania odcisków, a jedynie zbadanie możliwości zastosowania innego podejścia. W pracy przedstawiane są rożne metody porównywania kodów. Analizowane są metody porównywania w skali podobieństwa, niepodobieństwa oraz obu tych wskaźników jednocześnie. Wyniki uzyskanego rozwiązania prezentowane są na tle komercyjnego rozwiązania.}
    \vspace*{1\baselineskip}

    \noindent{\bf Słowa kluczowe}: Biometria, odcisk palca, kod odcisku
    \par
    \vspace{4\baselineskip}
	\noindent{\bf Distance, similarity and dissimilarity of images for fingerprints}
    \begin{center}
	{\large\bfseries Abstract}\par\bigskip
    \end{center}
    \vspace*{1\baselineskip}
    {
\itshape This thesis describes implementation and score of fingerprints compare. Described implementation uses code algorthm. Thesis shows advantages and disadvantages of presented methods. Crating a new software for fingerprints compare is not a target of this thesis. Thesis use several methods of code compare. Presented methods compare fingerprints using similarity, dissimilarity and both of this feature. Score of using algorith is compare with commercial solution.}
    \vspace*{1\baselineskip}

    \noindent{\bf Key words}: Biometric, fingerprint, finger code

\end{titlepage}

% ex: set tabstop=4 shiftwidth=4 softtabstop=4 noexpandtab fileformat=unix filetype=tex spelllang=pl,en spell:
